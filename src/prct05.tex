\documentclass{beamer}
\usepackage[spanish]{babel}
\usepackage[utf8]{inputenc}
\usepackage{graphicx}

\newtheorem{caracteristicas}{Características}

\title[Beamer]{Creación de Diapositivas con Beamer}
\author[Adrián Mendióroz]{Adrián R. Mendióroz Morales\\
Técnicas Instrumentales\\
Grado en Matemáticas\\
Universidad de La Laguna}
\date[15-03-2013]{15 de Marzo de 2013}

\usetheme{Madrid}
\usecolortheme[RGB={122,59,122}]{structure}

\begin{document}

\begin{frame}
 
\titlepage
 
\end{frame}

\begin{frame}
\frametitle{Índice}  
\tableofcontents[pausesections]
\end{frame}

\section{¿Por qué Beamer?}

\begin{frame}

\frametitle{¿Por qué Beamer?}

\begin{caracteristicas}

Para la creación de Diapositivas usamos Beamer porque posee ventajas importantes como por
ejemplo, que los comandos de \LaTeX también funcionan en Beamer, porque se crea un índice 
automático con enlaces a cada sección y subsección, porque el formato de salida es usualmente
PDF, el cual tiene una compatibilidad global, o porque (según veremos en la siguiente sección)
permite un sencillo uso de fórmulas matemáticas.

\end{caracteristicas}

\end{frame}

\section{Fórmulas Matemáticas}

\begin{frame}

\frametitle{Fórmulas Matemáticas}

\begin{block}{Objetivo}
 A continuación vamos a ver unos ejemplos de como se muestran las fórmulas matemáticas usando Beamer.
 
\end{block}

\begin{block}{Ejemplos}

\begin{itemize}
  \item
  \begin{scriptsize}
  \[ S_n=a_1 + a_2 + \cdots + a_n = \sum_{i=1}^n a_i \]
  \end{scriptsize}
  \pause

  \item
  \begin{scriptsize}
  \[ \int_{x=0}^{x=1} x\text{e}^{x^2} \]
  \end{scriptsize}
  \pause

  \item
  \begin{scriptsize}
  \[ y=\frac{x^2 + 3x + 1}{1 + x^2} \]
  \end{scriptsize}
  
\end{itemize}

\end{block}

\end{frame}
 
 \begin{frame}

\frametitle{Fórmulas Matemáticas}

\begin{block}{Ejemplos}

\begin{itemize}

  \item
  \begin{scriptsize}
  \[ lim_{x\to\infty} \left(x + \frac{1}{x} \right) \]
  \end{scriptsize}
  \pause
  
  \item
  \begin{scriptsize}
  \[ x + y^{2n + 2} = \sqrt{b^2 - 4ac} \]
  \end{scriptsize}

  
\end{itemize}

\end{block}

\end{frame}

\begin{frame}

\frametitle{Bibliografía}

\begin{thebibliography}{10}

\beamertemplatebookbibitems
\bibitem{Beamer}Beamer - Presentaciones en LaTeX
 {\tiny $http://www.dsi.uclm.es/personal/AnaMariaMartinez/webcurso/curso_archivos/LaTeX/8Beamer.pdf$}

\beamertemplatebookbibitems
\bibitem{Formulas}Creación eficiente de documentos con LaTeX
 {\tiny $http://www.alumnos.inf.utfsm.cl/~mmora/documents/presentacion-latex.pdf$}

\end{thebibliography}

\end{frame}


\end{document}
